\Chapter{Összegzés}

% TODO: Röviden értékelni kellene, hogy mi és hogy készült el a dolgozat megírása során.
% Ez sem kell, hogy túl hosszú legyen, és elég csak a végén megírni.

A dolgozat egy fejlesztői környezet fejlesztésének menetét mutatta be a Rust programozási nyelvben és nyelvhez.

A dolgozat megírása előtt hallottam már a Rust nyelvről, és érdekelt is voltam benne, 
viszont nem dolgoztam még vele.
Hasonlót mondhatok a GTK widgetkönyvtárról is -- bár használati szinten ismertem, 
mivel a Linux asztal-környezetem a GTK-n alapszik, használni még nem használtam.
Ugyan ez nem az első alkalom, hogy felhasználói felületet kellett készítenem egy programhoz,
ebből a szempontból jobban ismerem a Qt könyvtárát.

A program írása során sok problémába ütköztem a Rust sajátosságai miatt.
Ha le kellene írnom a tapasztalataimat a Rust-tal, akkor azt mondanám, hogy használatakor
a problémák az ismerkedés elején jelennek meg, majd ahogy a programozó egyre tapasztaltabb
a használatában, úgy a Rust-ban való programozás egyre egyszerűbb lesz --
ellentétben például a C++-szal, amit elkezdeni egyszerű, de a nyelv használata jelentősen megnehezedik,
ahogy a programozó belemélyed a mutatók, referenciák, és sablonok használatába.

Ez egyszerre egy pozitívum és egy negatívum.
Az első projekt elkészítése során sokan megrémülnek a nyelv látszólagos nehézségétől,
de ellentétben egy emberrel, a fordító "figyelme" nem lankad még a tizedik projektnél sem,
amikor már a tapasztalt felhasználó örömmel fogadja a fordító javaslatait.

Bár a Rust használata egyre könnyebbé válik, ahogy az idő telik, a GTK Rust binding használata pont az ellentéte ennek.
A dolgozat során sokszor említettem, hogy a GTK-s objektumok nem tisztelik a Rust szabályait,
és emiatt a GTK eredetű struktúrák helyes használata sokszor rejtélyes vagy logikának ellentmondó
(például a klónozható struktúrák).

Ezt a problémát nem segíti a \texttt{gtk-rs} láda dokumentációja.
A használati útmutatót régen frissítették, és a legtöbb bemutatott példa nem fut a jelenlegi verziókban,
az API dokumentáció pedig hiányos magyarázatokban, és néhol a magyarázat a C nyelvre utal, és annak használatát mutatja be.
Sőt, a GTK öröklődési szabályainak megtartása érdekében a fontosabb metódus hívásokat, amiket a felhasználó kereshet,
nem a megadott struktúrában kell keresni, hanem valamelyik tulajdonságban, amit implementál.
Gyakori az, hogy egy \texttt{Valami} struktúra 
leggyakrabban használt metódusát a \texttt{ValamiExt} tulajdonságban kell keresni.

Ez természetesen nem rossz, vagy elkerülendő programozási módszertan.
A \texttt{File\-ChooserDialog} logikusan implementálja a \texttt{DialogExt}, \texttt{FileChooserExt},
\texttt{WidgetExt} és más tulajdonságokat, és ezeken osztozik más struktúrákkal.
Ha a felhasználó tudja, hogy mit keres, akkor ez a megoldás tökéletes.
Együtt az előbb említett hiányos dokumentációval, és a félrevezető példákkal viszont zavaró és átláthatatlan.

Sajnálatos módon továbbá az RLS-t nem volt időm bekötni a programba.
Alábecsültem a programhoz szükséges tapasztalatot, illetve az időt, ami alatt ezt a tapasztalatot megszerezhetem.
Egy IDE sok mozgó részből áll, és ezt kezdetben nem láttam át.

Ennek ellenére sokat tanultam a fejlesztői környezetek felépítéséről, hogy hogyan működnek,
és ha még egyszer neki kellene vágnom ennek a projektnek, akkor úgy gondolom,
hogy sikerülne felépíteni egy igazi Rust IDE-t.

% TODO: A felsorolt tapasztalatok mindenképpen jók. Érdemes lenne kicsit objektívabban összefoglalni majd a dolgozat eredményeit, a későbbi fejlesztési lehetőségeket.
