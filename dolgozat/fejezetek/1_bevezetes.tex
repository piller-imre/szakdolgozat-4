\Chapter{Bevezetés}

\Section{Integrált Fejlesztői Környezetek (IDE-k)}

Az integrált fejlesztői környezet, vagy röviden \emph{IDE} (az angol \emph{integrated development environment} megnevezésből) 
egy olyan szoftver, ami elősegíti a felhasználó programozót más szoftverek fejlesztésében. 
Egy minimális IDE tartalmaz legalább egy forrásfájl-szerkesztőt, automatikus fordítási lehetőségeket, 
és egy debuggert, de a mai IDE-k nagyszámú egyéb szolgáltatásokat is nyújtanak a felhasználónak, 
például intelligens kódkiegészítést már akkor, mikor a felhasználó csak éppen belekezdett egy kulcsszó írásába, 
támogatást külső verziókövető szoftverekhez, vagy fordítást nem igénylő kódellenőrzést a projektben.

Így nyilvánvaló, hogy miért előnyös egy IDE létezése egy megadott nyelvhez: 
a fentebb felsorolt szolgáltatásokat egy IDE nélkül a programozó vagy csak különféle, egymásba nem feltétlenül integrált szoftverekkel érhetné el 
(például: a nyelvhez specifikus fordító, és a programozó által használt tetszőleges szövegszerkesztő), 
vagy nem feltétlenül érhetné el egyáltalán (intelligens kódkiegészítés).

A szövegszerkesztőknél meg kell említenünk, hogy ma már a legtöbbjük intelligens eléggé ahhoz, 
hogy egy IDE funkcióit maguk is szolgáltatni tudják. Ilyen szövegszerkesztőket, 
és hogy hogyan támogatják a szoftverfejlesztést Rust-ban, a következő részben, 
\myaref{sec:ides} alfejezetben fogjuk megtekinteni, illetve a dolgozat elmagyarázza, hogy ezek a megoldások miért nem elégségesek.

\Section{IDE-k Rust-ban}
\label{sec:ides}

Mint azt már említettük, sok szövegszerkesztő ajánl IDE-szerű szolgáltatásokat Rust-hoz. 
A Rust közössége ezeket le is jegyzi, és egy nyilvánosan elérhető, 
\emph{Are we (I)DE yet?}\cite{ideyet} című weboldalon összegezi.

Mint azt a weboldalon láthatjuk, támogatás a Rust-hoz elérhető a legelterjettebb szövegszerkesztőkben, 
mint az Atom, Sublime Text, Visual Studio Code, stb. 
A legutóbbi kiemelendő, mivel a táblázat alapján látszólag minden meg van benne.

A szövegszerkesztők egyszerre több nyelvhez tudnak ilyen szolgáltatásokat nyújtani. 
Bár néhány szolgáltatás a szövegszerkesztőtől magától függ csak, a legtöbbhöz segítséget nyújt egy \emph{nyelvszerver} (\emph{language server}). 
A nyelvszerver egy olyan szoftver, ami a \emph{nyelvszerver protokoll}\cite{lsp} alapján információt nyújt IDE-knek, szövegszerkesztőknek, és egyéb programoknak a hozzá tartozó nyelvről.

Mivel a szövegszerkesztők bármilyen nyelvhez támogatást nyújthatnak egy közös protokoll által, így nyilvánvaló, hogy bármilyen szolgáltatás, ami a protokollon kívül esik, nem lesz elérhető egy nyelvhez sem. 
Van továbbá néhány funkció, amit viszont a szövegszerkesztőtől magától nem várhatunk el -- 
például egy ablaképítő, amivel grafikusan szerkeszthetjük az alkalmazás ablakait, messze áll a szövegszerkesztéstől, és nem is lenne értelme a létezésének egy csupasz, nyelvtámogatás nélküli szövegszerkesztőben. 
Ne feledjük, hogy a nyelvtámogatás maga csak egy mellékfunkció; 
a szövegszerkesztő attól függetlenül szöveget még mindig tud szerkeszteni, de ha ablaképítőt is tartalmazna egy ilyen szoftver, akkor a szoftver fele használhatatlan, és így értelmetlen lenne anélkül, hogy magunk ki nem bővítenénk azt.

Egy másik megoldás, amit észrevehetünk az oldalon, az az, hogy a Rust támogatást meglévő IDE-kbe építjük bele, mint pl. Eclipse, IntelliJ IDEA, vagy Visual Studio. 
Bár azt gondolnánk, hogy ez közelebb áll a célunkhoz, valójában néhány szempontból pontosan annak az ellentéte igaz: ilyen esetben a Rust mint másodrendű nyelv kerül bele az IDE-ba -- 
Eclipse és IntelliJ IDEA esetén a Java alá, Visual Studio esetén a C\# alá főként. 
Észrevehetjük, hogy támogatás szempontjából nem a nyelvi támogatás lett komplexebb, hanem a szoftver maga; 
a Rust szempontjából egy 'felturbózott' szövegszerkesztővel dolgozunk, és nem egy IDE-vel.

Legutoljára láthatjuk, hogy a Rust-hoz bizony készült már egynéhány IDE, ezek a \emph{Ride}\cite{ride} és a \emph{SolidOak}\cite{solidoak}. 
Sajnálatos módon ezek sem felelnek meg a követelményeinknek: nem csak, hogy mindkettő gyérebb a funkcionalitás oldalon, mint egy mezei szövegszerkesztő, de továbbá mindkét projekt halott is. 
A szakdolgozat írása idejében a Ride legutolsó módosítása a projekt megszüntetésének bejelentése, és nem ajánlott produkciós környezetben, a SolidOak Github oldala pedig archiválva lett.

\Section{Egy Rust IDE fejlesztése}

Az előző elemzésekből kitűnhet, hogy nyelvi szinten a Rust készen áll már akár most is egy Rust-centrikus IDE-re. 
%Továbbá, a Rust Forge oldalán látható, hogy tervekben áll\footnote{\texttt{https://forge.rust-lang.org/ides.html}} egy hivatalos daemon fejlesztése, ami tovább segíteni a leendő IDE-k munkáját egy Rust projekt kezelésében, és a lefordító módosítása, hogy támogassa az inkrementális és lusta fordítást, és hogy különféle hibák ellenére is végigmenjen a fordítási folyamaton.

Így a feladat a következő: létrehozni egy olyan IDE-t, ami a Rust nyelvet elsőrendű nyelvként kezeli, és annak előnyeire játszik. 
Fontos, hogy az IDE kényelmessé tegye a nyelvben való programozást, így a fejlesztés során gyakori kell, hogy legyen az iterálás a felhasználó munkamenetén. 
Fejleszteni kell majd olyan szolgáltatásokat, amik könynyebbé teszik a nyelv sajátosságainak megértését, például az utalások élettartamának nyomonkövetését 
(ezeket a sajátosságokat majd bővebben bemutatjuk \myaref{sec:rust} fejezetben), és amik figyelmeztetnek a gyakori Rust programozási hibákra.

% TODO: Itt egy olyan leírásnak kellene majd szerepelnie, ami felhívja az olvasó figyelmét, hogy
% - milyen fontos, hogy legyen ilyen IDE,
% - jelenleg még nincs ilyen,
% - a feladat elvégzése nem triviális.

% TODO: Címszavasan bemutatásra kell, hogy kerüljenek a felhasznált technológiák.

% 1-2 oldal elég