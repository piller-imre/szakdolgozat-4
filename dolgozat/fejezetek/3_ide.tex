\Chapter{Integrált fejlesztőkörnyezet}

% TODO: Fel kell sorolni, hogy milyen részekből épül fel egy fejlesztőkörnyezet, milyen funkciókat tartalmaz!

% TODO: Ki kell külön térni a bővíthetőségére, személyreszabhatóságára!

% TODO: Specifikálni kell az elkészítendő szoftvert!

\Section{Követelmények}

Ahhoz, hogy egy felhasználó-barát fejlesztőkörnyezetet készítsünk, fontos, hogy ne feledjük, hogy egy \emph{környezetet} készítünk: 
azaz, az IDE-nek magában kell, hogy foglaljon mindent, ami szükséges lehet szoftverfejlesztéshez Rust-ban, legalább a standard könyvtár kezelése során. 
Ebből következőleg az IDE fő célja az kell, hogy legyen, hogy a felhasználónak a lehető legkevesebbet kelljen az fejlesztői környezetét elhagyni -- 
ha valamit szükséges neki megtenni, akkor azt fontos, hogy az IDE-n keresztül, illetve annak segítségével, tudja véghezvinni. 
Ilyen környezeten kívüli elkerülendő dolgok például a terminál illetve parancssor használata fordításhoz, 
a dokumentáció megtekintése \texttt{cargo doc}-on keresztül, vagy egy külső verziókövető szoftver használata.

Nyilván a fejlesztőkörnyezet célja nem az, hogy ezeket a segédeszközöket leváltsa -- sőt, az IDE maga ezekkel fog kommunikálni! 
A specializált szoftver elkerülhetetlenül többre lesz képes, mint a generalizált IDE, így a magasabb képességű programozó igényeit nem feltétlenül fogja kiszolgálni csak az IDE maga, 
viszont az átlagos programozó szempontjából célszerű az, hogy a fejlesztés a lehető legkevesebb gondolatbeli kontextus-váltással járjon.

Az IDE céljait behatárolva felsorolhatjuk, hogy mire kell, hogy képes legyen a fejlesztőkörnyezet. 
A szükséges funkciók a következők:

\begin{itemize}
	\item A forráskódok szerkeszthetőek legyenek az IDE-n belül. Ez talán egy átlagos IDE legalapvetőbb követelménye. 
	Elégséges az, hogy ha egyszerre egy fájlt tudunk megnyitni. 
	Fontos itt a Rust nyelv kulcsszavainak kiemelése színezéssel és félkövéresítéssel, az automatikus zárójel-kiegészítés (ha \texttt{(}-et kezdünk, akkor a program automatikusan beillessze a megfelelő \texttt{)}-et is), 
	illetve -összekapcsolás (a program ki kell, hogy emelje, hogy egy zárójelnek hol a párja).
	
	\item Kapcsolódva a fenti ponthoz, az IDE-nek szükséges kódjavaslatokat feltenni a felhasználónak, mind új kód írásakor, és régi kód javításakor. 
	Mint a legtöbb nyelvben, a Rust-ban is létezik egy \emph{nyelvszerver}, a \emph{Rust Language Server}\cite{rls}. 
	A nyelvszerver szerepe pontosan ez, így szükséges lesz kommunikáció kiépítése az \emph{RLS}-sel. 
	Ha valamiért ez nem sikerülne, akkor mi magunk is megoldhatunk egy hasonló funkciót a fordítóprogram és a \emph{racer}\cite{racer} segítségével -- 
	az RLS maga is ezt csinálja. Ennek ellenére preferáltabb az RLS-sel dolgozó megoldás.\label{rls}
	
	\item Az IDE tudjon elvégezni lintelést, illetve projektformázási javaslatokat. 
	Az előbbi megvalósítható a \texttt{clippy}\cite{clippy} program segítségével, az utóbbi a \texttt{rustfmt}\cite{rustfmt} programmal.
	
	\item Szükséges, hogy a felhasználó tudja a projekthez kapcsolatos fájlok kezelését az IDE-n belülről. 
	Ez alatt az is értetendő, hogy a felhasználó létre tudjon hozni cargo projekteket az IDE-n keresztül, illetve hogy be tudjon tölteni régi cargo projekteket az IDE-be. 
	Fontos, hogy csak az számítson, hogy a betöltött projekt cargo-val lett létrehozva, ne az, hogy az IDE által. 
	Az IDE tárolhat információkat a projektról egy rejtett mappában, de ne legyen kötelező a projekt módosításához. 
	Továbbá, az IDE-n belül fontos, hogy legyen a projekt fáljrendszerének egy fa-struktúrás reprezentációja, ami segítségével a felhasználó tudjon refaktorálást végezni a fájlokon, vagy a cargo követelményeit betartva hozhasson létre fájlokat 
	(A Rust 2016-os verziója például megköveteli, hogy a modulmappák 'fő' forrásfájlja a \texttt{mod.rs} nevet kapja).
	
	\item Az IDE tudja a projektet lefordítani debuggoláshoz és kiadáshoz, tudja a projekt tesztjeit lefuttatni, és tudja a projekt problémáit leellenőrzni. 
	Ezek viszonylag egyszerű cargo parancsmeghívások, a fontos rész az, hogy ezeket a felhasználó egy gombnyomással vagy billentyűkombinációval meg tudja hívni.
	
	\item Az IDE tudjon dokumentációt szolgáltatni a felhasználónak a kód megadott részeiről. 
	Ez a funkció valójában három különböző szinten kell, hogy létezzen:
	\begin{enumerate}
		\item A kódszerkesztés közben az IDE-nek magyarázatot kell szolgáltatni az éppen fókuszban lévő kódrészletről. 
		Funkció esetén a funkció paramétereiről és visszatérési értékeiről, struktúra esetén a publikus mezőkről, illetve mindkettő esetén egy rövid leírást kell szolgáltatnia róla.
		
		\item A felhasználó külön kérésére az IDE-nek meg kell nyitnia a megadott kódrészlethez kapcsolódó dokumentációt, ha létezik. 
		Rust-ban szokás \texttt{cargo doc} segítségével weblap-alapú dokumentáció készítése, így az IDE-nek a háttérben kötelező a dokumentációt legenerálnia, és a megadott weblapot renderelnie magán belül.
		
		\item Ha a felhasználó ezen kívül szeretne dokumentációt olvasni, akkor ezeket az IDE jelenítse meg a felhasználó megadott webböngészőjében. 
		Ezt a lépést tekinthetjük túl specializáltnak az IDE-hez.
	\end{enumerate}
	
	\item Az IDE-nek szolgáltatnia kell alapvető verziókövetési támogatást. 
	Mivel a git a legelterjettebb verziókövető alkalmazás, ezért az IDE-nek tudnia kell git által változtatásokat feltölteni, letölteni, commit-olni, ágakat létrehozni.
	
	\item Végül, ha szükséges, az IDE legyen bővíthető. 
	Hasonlóképpen a \emph{Visual Studio}-hoz, legyen lehetőség saját parancsok létrehozására, amikkel a felhasználó tetszőleges programokat hívhat meg. 
	Ezeket a kényelem kedvéért az IDE beillesztheti egy menübe, így a parancs meghívása nem igényelne, csak néhány kattintást.
\end{itemize}

\Section{Specifikáció}

\SubSection{Külső követelmények}

Mivel egy grafikus szoftvert szeretnénk létrehozni, ezért legelőször egy widget könyvtárat kell megállapítani alapkövetelményként.
A Rust widget-ökoszisztémája még nem fejlett, ezért nincs nagyon sok választási lehetőség: főként a GTK és a Qt.
Végül is a GTK widget könyvtár mellett döntöttem, mert sokkal érettebb a Qt binding könyvtárakhoz képest.

Ha GTK, akkor szeretnénk a \texttt{GTKSourceView} projektet is függőséggé tenni, 
mivel ez lehetőséget ad a könnyű kódszerkesztő létrehozására -- ez egy kibővített \texttt{GTKTextView} pontosan kódkezelésre,
és támogat viszonylag sok programozási nyelvet, köztük a Rust-ot is.
Emellett automatikusan számolja a sorokat, kiszínezi a megadott nyelv kulcsszavait, 
kezeli a fehérhelyeket és behúzásokat kódnak megfelelően, és sok más.
Fontos, hogy a \texttt{GTKSourceView} támogatás csak a Rust-GTK \texttt{v3\_22} verziójától elérhető.

A \texttt{GTKSourceView}-ben lévő szöveg kirajzolásához szükséges lesz a \texttt{Pango} könyvtár behívása.
Egyelőre itt további követelmény vagy érdekesség nincs -- viszont, ha később azt szeretnénk, 
hogy a dokumentációt ne külső webböngészőben jelenítsük meg, hanem a programon belül, akkor több haszna is lehet ennek a könyvtárnak.

A billentyűkombinációkhoz szükséges lesz behívni a \texttt{GDK} könyvtárat, hogy az ablak által kapott
billentyű-lenyomás jelzéseket elkaphassuk, feldolgozhassuk, és értelmezhessük.

Végül, szükséges lesz egy könyvtár, aminek segítségével össze tudunk hasonlítani két szövegblokkot.
Erre egy megoldás a SHA kriptográfiai függvény, aminek egy SHA-3 implementációját a \texttt{tiny-keccak} könyvtár szolgáltatja.

Összességében, a következő Rust "ládákra" (Rust csomagokra) lesz szükségünk a minimális funckiókhoz:

\begin{itemize}
	\item \texttt{gtk} (\texttt{v3\_22} funkcióval legalább)
	\item \texttt{gdk}
	\item \texttt{pango}
	\item \texttt{sourceview}
	\item \texttt{tiny-keccak}
\end{itemize}

Ezen túl, mint már azt \myaref{rls} részben is említettük, a fejlesztőkörnyezeti funkciókhoz szükségünk lesz arra,
hogy a Rust nyelvszerverével, az RLS-sel kommunikáljunk.
Lehetne az RLS-t meghívni a kódban könyvtárként, de az LSP-n\cite{lsp} alapuló nyelvszerverek lehetőséget
adnak a standard bemeneti és kimeneti csatornákon való kommunikációra, 
így elégséges az, ha gyermekfolyamatként meghívjuk az RLS-t.
A csatornák kezelésére a Rust standard könyvtára lehetőséget ad az \texttt{io} modul \texttt{stdin()}
és \texttt{stdout()} metódusokkal.
Az RLS továbbá követelményként a Rust forráskódját igényli a javaslatok és dokumentáció megjelenítéshez
a standard könyvtárban lévő struktúrákhoz és metódusokhoz.

Az LSP alapú nyelvszerverek JSON-RPC\cite{json-rpc} távoli-eljáráshívás protokoll alapján kommunikálnak
a kliens programokkal, ezért szükség lesz egy JSON szerializáló és deszerializáló könyvtárra is.
Erre a legelterjettebb megoldás a \texttt{serde} könyvtár, vagyis a mi esetünkben a \texttt{serde-json} alkönyvtára.

Ahhoz, hogy az RLS-t használni tudjuk egy Rust projekten, szükségünk lesz a Rust projektek készítésére is.
Ehhez \myaref{cargo_mention} részben is említett Cargo csomagkezelőt fogjuk használni.

Mind az RLS-t, mind a Cargo-t gyermekfolyamatként kell majd meghívnunk.
A Rust standard könyvtárában, a \texttt{process} modulban létező \texttt{Command} struktúra
lehetőséget az a gyermekfolyamatok létrehozására, illetve azok standard bemenetének és kimenetének kezelésére.

Így a fejlesztőkörnyezet szemszögéből az alábbiakkal bővül ki a követelménylista:

\begin{itemize}
	\item RLS
	\begin{itemize}
		\item Rust forráskódja
	\end{itemize}
	\item Cargo
	\item \texttt{serde}(\texttt{-json})
\end{itemize}

\SubSection{A program felépítése}

A program központi eleme egy \texttt{App} struktúra lesz. 
Ez tárolja majd a referenciát a program egyetlen ablakára (\texttt{Window}), a fejlécre (\texttt{Header})
és a tartalomra.
A tartalom egy hasonló struktúra az \texttt{App}-hoz, ami egy tárolót és magát a szövegszerkesztő \texttt{SourceView}-et tartalmazza.

Mivel fájlokat akarunk kezelni, ezért szükségünk lesz a Rust standard könyvtárából a \texttt{File} struktúrára
az \texttt{fs} ("filesystem") modulból.
A fájlkezelésnél szintúgy ajánlatos az, hogy figyeljük, hogyan érjük el a fájlt.
Vagy egyszerre csak egy író írhat a fájlba, vagy egyszerre több olvasó olvashat belőle.
Erre a standard könyvtár \texttt{sync} moduljában létezik az \texttt{RwLock} zár ("reader-writer lock"),
ami pontosan erre a célra lett létrehozva.
\texttt{Mutex} itt nem lenne megfelelő, mivel egy \texttt{Mutex} zár nem tesz különbséget írók és olvasók között,
így akkor is lezárná az erőforrást, ha biztosan tudjuk, hogy az azt éppen elérő olvasó nem módosítaná.

Illetve a felesleges fájlkiolvasások megspórolása érdekében nem a fájl tartalmával fogunk dolgozni főképp.
Létrehozunk egy \texttt{ActiveMetadata} struktúrát, ami magában tárolja a fájl elérési helyét,
illetve a legutóbb mentett szöveg keccak hash szummáját.
A fájl valós tartalmával tulajdonképpen soha nem hasonlítunk össze.
Ez nem feltétlenül baj, csak fejlettebb szövegszerkesztőknél található olyan funkció, ami felismeri
egy külső fájl változását, például \textit{Visual Studio Code}.
Ha mégis probléma, akkor a jövőben kiküszöbölhetjük egy fájl-figyelő rendszer bevezetésével \textit{Visual Studio Code} módjára,
vagy csak figyelmeztethetjük a felhasználót mentéskor, hogy a fájlt külső programok módosították utolsó mentés óta,
és felajánlhatunk felülírási és elvetési opciókat.

Szükségünk lesz továbbá megint a \texttt{sync} modulból az \texttt{Arc} struktúrára.
Ez lehetőséget ad az atomikus referencia-számolásra.
Számunkra ez azért szükséges, mert az \texttt{App} jelenlegi fájl \texttt{ActiveMetadata}-ját
a grafikus felület legtöbb elemének el kell érnie.
A \textit{Mentés}, \textit{Mentés Másként} és \textit{Megnyitás} gombok mind módosítják legalább a keccak-szummát,
de például az utóbbi kettő esetében a fájl elérési helyét is (hacsak a felhasználó nem ugyanabba a fájlba ment,
vagy ugyanazt a fájlt nyitja meg).
Ezelőtt \myaref{subsec:ownership_mentioned} részben már említve volt a Rust egyedi birtoklás rendszere.
Ez megakadályozza, hogy mozgatás után elérjük a mozgatott erőforrást, és a gomblenyomás jelzés
továbbmozgatná a jelenlegi fájl metaadatokat a meghívott mentés vagy megnyitás metódusba
(bár igazából a birtoklás rendszer miatt már előbb is megbukna a kód, mert a jelzés-meghívott metódus
összecsatolás maga szükségessé tenné a metaadat-erőforrás mozgatását).
Az \texttt{Arc} struktúra \texttt{clone()} metódusa lemásolja az \texttt{Arc} struktúrát,
viszont a létrejövő struktúra ugyanarra az erőforrásra mutat, és az erőforrásra mutató \texttt{Arc}-ok
száma nyilván van tartva -- ha már nem mutat semmi az erőforrásra, akkor elpusztul az erőforrás maga is.
Az \texttt{Arc} működése tekinthető hasonlónak a mutatókhoz szemétgyűjtő mechanizmussal rendelkező nyelvekben.

Az \texttt{Arc} struktúrát még egy másik helyen használjuk, a teljes képernyős mód kezelésénél.
A használati ok megegyezik az előzővel, annyi különbséggel, hogy itt most egy \texttt{AtomicBool} struktúrát csomagolunk be.
Az \texttt{AtomicBool} egy szál-biztonságos \texttt{boolean} változó.

Az alfejezet elején röviden említett grafikus felületi elemek esetén nincs sok tennivaló.
Itt inkább, mint például a \texttt{Window} és a \texttt{Header} esetében, az elemek elhelyezkedése és viselkedése az,
amit beállítunk.
A \texttt{Window} csak a \texttt{Header} és a már előbb említett tartalom (továbbiakban \texttt{Content})
elemeket tartalmazza.
A \texttt{Header} tartalmazza a \textit{Megnyitás}, \textit{Mentés} és \textit{Mentés Másként} gombokat,
bár mint azt már említettük, a gombok lenyomásához az \texttt{App} maga rendel akciókat.

A \texttt{Content} rész egy kicsivel érdekesebb.
Ez tartalmaz egy tárolót és a \texttt{SourceView}-et (igazából egy struktúrát ami azt becsomagolja -- 
később látjuk, miért).
Itt tulajdonképpen a tároló eleinte felesleges, hiszen csak egy dolgot tárolunk benne, de későbbi kibővítések
érdekében benne hagyjuk egyelőre.
A tárolóban egy \texttt{Source} struktúra magában foglal három elemet: egy \texttt{ScrolledWindow} GTK widgetet,
ami a második elemet, a \texttt{SourceView}-t tárolja, így görgethetővé téve azt, 
illetve a \texttt{sourceview} modulból egy \texttt{Buffer} elemet, ami nevéből adódóan szövegpuffer a \texttt{SourceView}-nek.
Bár lehetőség van arra, hogy a \texttt{SourceView} elemet pufferrel együtt hozzuk létre,
kinyerni a puffert egy így létrehozott \texttt{SourceView}-ből nehézkes,
és a mi esetünkben van értelme külön tárolni is, mert mint azt már említettük, a mentés lehetőségek
a meglévő szöveg keccak hash-jét használják fel annak eldöntésére, hogy a fájl módosult-e.

Végül érdemes még kiemelni a fájl mentés és betöltés folyamatát.
Ez egy \texttt{File\-Chooser\-Dialog} meghívásával történik, viszont itt belefutunk egy gyakori
problémába a GTK Rust binding-jaival: mivel nem natív a GTK könyvtára,
ezért gyakran a Rust szokásainak ellentétesen, vagy azokhoz képest furcsán viselkedik.
Egyik gyakori eredménye ennek az, hogy valójában a GTK-s elemek csak mutatók a Rust alatt futó
tényleges GTK-s elemekre\cite{glib:no_need_for_borrows}, így az objektumok klónozása olcsó,
és a metódushívásaik soha nem igényelnek mutálható kölcsönzéseket.
Ebben az esetben a \texttt{FileChooserDialog} az egyik olyan GTK Rust elem, ami soha nem pusztul el meghívása után,
így szükséges nekünk implementálni hozzá a \texttt{Drop} tulajdonságot, de ha már úgy is becsomagoljuk
saját struktúránkba, akkor ketté is választhatjuk \texttt{SaveDialog} és \texttt{OpenDialog} felugró ablakokká.
Ez megspórolja nekünk az ablakok újbóli szerkesztését amikor meg kell hívnunk egy \texttt{FileChooseDialog} ablakot.

Miután a grafikus felület elkészült, a program kibővíthető az IDE lehetőségekkel.
A \texttt{Command} parancshívás-építő struktúra segítségével kommunikálhatunk a programon kívüli világgal.
A struktúrát főképp két helyen fogjuk használni, az \texttt{App} illetve a \texttt{Content} struktúrákban.
A Cargo parancsok meghívásának variációit tárolhatjuk az \texttt{App} alatt egy listában,
mivel a parancsmeghívás-építő nem kötelezi a parancs azonnali meghívását az építés lefolyása után,
és egy megépített parancs többször is meghívható.
Továbbá, a Cargo parancshívások argumentumlistái nem változnak túlságosan nagy mértékben,
hogy azt szükséges legyen minden esetben újraépíteni meghívás előtt memóriatakarékosság érdekében.

A következő alapvető Cargo parancsokat érdemes implementálni:

\begin{itemize}
    \item \verb+cargo new --bin [név]+: új bináris Cargo projekt készítése. 
    A megadott név lesz a projekt kezdetleges neve, és egy ilyen nevű almappába kerülnek a minimális Rust projekt fájlok,
    mint pl. \texttt{Cargo.toml} és \texttt{src/main.rs}.
    \item \verb+cargo new --lib [név]+: új Cargo könyvtár projekt készítése.
    Hasonlóan működik a fenti parancshoz, de a \texttt{main.rs} fájl helyett egy \texttt{lib.rs} fájlt hoz létre.
    \item \verb+cargo build+: hibakereső módban lefordítja a projektet.
    Ha szükséges, letölti és lefordítja a kellő követelményeket.
    \item \verb+cargo build --release+: kiadás módban fordítja a projektet.
    \item \verb+cargo check+: leellenőrzi a projektet és a követleményeit, hogy vannak-e bennük hibák.
    \item \verb+cargo test+: leellenőrzi a projektben megadott teszteket.
    \item \verb+cargo bench+: leméri a \verb+#[bench]+ jelzővel ellátott tesztek lefutási idejét.
    \item \verb+cargo doc+: lefordítja a projektben lévő (publikus) dokumentációt.
    \item \verb+cargo doc --open+: lefordítja és megnyitja a dokumentációt.
\end{itemize}

Az RLS-es való kommunikációt a \texttt{Content} struktúra fogja kezelni.
A \texttt{Source} struktúra egy forrásfájlt kezel, és bár általában a jelenlegi fájlhoz szeretnénk
nyelvszerver segítséget, a nyelvszervernek tudnia kell az egész projektről a segítség ajánlásához.
A gyermekfolyamat létrehozásakor az \texttt{.stdout(Stdio::piped())} és az \texttt{.stdin(Stdio::piped())}
építési módosítókat megadva a gyermekfolyamat be- és kimeneti csatornái elérhetőek a \texttt{child.stdin}
és a \texttt{child.stdout} struktúrákon keresztül, és ezekből szabadon lehet írni és olvasni a program futása alatt.

A kommunikáció maga a \texttt{serde-json} könyvtár segítségével történik.
Ez lehetőséget ad struktúrák szerializációjára és deszerializációjára, amit kihasználhatunk úgy,
hogy az LSP specifikáció\cite{lsp-spec} által definiált gyakori üzeneteket struktúrákként implementáljuk --
ezzel elkerülve az emberi hiba lehetőségét is a kódban.
Ha mégis szükségünk lenne egy manuálisan előállított JSON csomagra, akkor a könyvtár lehetőséget ad 
arra is a \texttt{json!()} makróval.