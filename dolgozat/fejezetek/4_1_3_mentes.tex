\SubSection{Mentés metódus és a fájlválasztó felugró ablak}

\subsubsection{\texttt{ActiveMetadata}}

Ahhoz, hogy a fájlmentést létrehozzuk, szükségünk lesz először egy rendszerre, ami megállapítja,
hogy az éppen szerkesztett fájl, és az, ami a háttértáron szerepel, ugyanaz-e.
Erre \myaref{active_metadata} részben említett \texttt{ActiveMetadata} struktúrát készítjük el,
és használjuk fel.

A \texttt{main.rs} fájlunk mellé behozunk egy \texttt{state.rs} fájlt, amiben a betöltött
fájl állapotát kezeljük.

\lstinputlisting[firstline=3, lastline=29, language=Rust]{./kodreszletek/02_save.rs}

A legtöbb metódus ebben a fájlban magától értetődő.
Az \texttt{ActiveMetadata} struktúrának két adattagja van, egy útvonalpuffer,
ami a fájl elérési helyére mutat, illetve egy 64 elem hosszú, 8-bites nemnegatív egész számokból álló tömb,
ami egy 512 rátás Keccak algoritmus eredményét tartalmazza, a fájl tartalmának bemenetével.

Említésre méltó a \verb+get_dir()+ metódus, ami kicsit bonyolultabb a Rust sajátosságai miatt.
A \texttt{Path} és a \texttt{PathBuf} típusok kapcsolata hasonló a \texttt{str} és \texttt{String}
típusokéhoz: az előbbi nem mutálható, még az utóbbi igen.
A \texttt{PathBuf} \texttt{parent()} metódusa egy \texttt{Option<Path>} eredménnyel tér vissza --
ez azért van, mert hasonlóképpen a szövegkezeléshez, itt is érdemesebb nem mutálható adattípusokkal
visszatérni, ahol lehetséges.
Az eredmény útvonal továbbá egy \texttt{Option}-ba van becsomagolva, mivel a metódus a programon
kívüli állapotról tér vissza információval, és annak helyességét nem tudja biztosítani
(például, a program indítása és a metódus meghívása között elvesztettük az olvasási jogot a fájl
könyvtárához, vagy valami kiszámíthatatlan hiba történt).
Ezen kívül kötelesek vagyunk nem félrevezetni a \verb+get_dir()+ metódus felhasználóit,
ezért nem csomagolhatjuk ki az útvonalat az \texttt{Option}-ből --
ha hiba történt, akkor azt a meghívó lekezeli.
Viszont, egy \texttt{Option}-ön belül a \texttt{Path}-ot átkonvertálhatjuk \texttt{PathBuf}-ra 
egy \texttt{map()} meghívás segítségével, ami kicsomagolás nélkül meghívja az általunk megadott
metódust a belső értékre.
Ha pedig a belső érték \texttt{None} (a Rust-ban a legközelebbi érték a \texttt{null}-hoz),
akkor semmi nem történik, és a \texttt{None} érték hiba nélkül továbbmegy.

\subsubsection{A felugró ablakok}

Ahhoz, hogy menteni tudjunk, továbbá el kell készítenünk a fájlválasztó felugró ablakokat.
Mint azt már \myaref{file_chooser_dialog} részben említettük, bár csak a \texttt{Drop} tulajdonságot
kellene implementálnunk a \texttt{FileChooserDialog}-hoz, e helyett kihasználhatjuk ezt a lehetőséget,
és definiálhatunk két különböző felugró ablakot --
egyet a nyitásra, és egyet a mentésre.

Így létrehozunk egy \texttt{ui/dialog.rs} fájlt:

\lstinputlisting[firstline=33, lastline=62, language=Rust]{./kodreszletek/02_save.rs}

A tömörség kedvéért a \texttt{SaveDialog} metódusa nincsenek itt feltüntetve.

Mivel mindkettő felugró ablak struktúra csak egy \texttt{FileChooserDialog}-ot csomagol be,
ezért itt a nevesített adattagos definíció helyett rendezett listákat (tuple) használunk
a struktúrák definíciójára.
Az ilyen definíció esetén az adattagok névtelenek, és elérésük a definíció sorrendje alapján történik,
nullától kezdve.
Ezt láthatjuk is a \texttt{run()} metódusban, ahol a becsomagolt \texttt{FileChooserDialog}-ot futtatjuk,
illetve annak a felhasználó által kiválasztott fájlját kérjük le a \texttt{self.0} megnevezéssel.

Mint azt már említettük, a \texttt{FileChooserDialog} háttérben maradása egy olyan tulajdonság,
ami nekünk nem szükséges, ezért szeretnénk implementálni a \texttt{Drop} tulajdonságot hozzá.
Ez könnyen megtehető, most, hogy már van két becsomagoló struktúránk is.

\lstinputlisting[firstline=64, lastline=74, language=Rust]{./kodreszletek/02_save.rs}

A tulajdonságok implementálása hasonló más nyelvek interfész-implementálásához, szintaktikától eltérve.
A \texttt{Drop} implementálásával kötelezőek vagyunk a tulajdonság által megadott minden metódust implementálni.
Ebben az esetben a \texttt{Drop} csak egyetlen egy metódust definiál, így nincs sok tennivalónk.

Az, hogy miért szükséges a \texttt{Drop} tulajdonságot implementálni, a Rust kölcsönzés-ellenőrző
rendszerének működésén alapszik.
Más nyelvekben vagy a szemétgyűjtő rendszer pusztítaná el a fájlválasztó felugró ablak forrásait
(mint a Java vagy a C\#), vagy pedig a programozó maga (\texttt{delete} a C++-ban).
A Rust-ban akkor szabadulnak fel erőforrások, ha a birtokosa a hatáskörén kívülre esik.
A \texttt{Drop} tulajdonság \texttt{drop()} metódusa pontosan ekkor lép életbe,
ezért ez a tökéletes alkalom arra, hogy a Rust kölcsönzés-ellenőrző rendszerén kívül eső
erőforrásokat felszabadítsuk -- mint például a GTK-s erőforrásokat a mi esetünkben.

Utolsó lépésként a mentés funkció elkészítéséhez létrehozunk kettő segédmetódust,
és ezeket a \texttt{ui/misc.rs} fájlba helyezzük.
Ezek a 
\begin{itemize}
    \item \verb+set_title(headerbar: &HeaderBar, path: &Path)+,
    ami beállítja a fejléc címét a megadott fájl elérési útvonalára,
    amit mi az éppen szerkesztett fájl kiíratására fogunk használni, illetve a
    \item \verb+get_buffer(buffer: &Buffer) -> Option<GString>+,
    ami visszatér egy megadott puffer teljes tartalmával.
\end{itemize}
Ezek csak kényelmi metódusok, és implementációjuk triviális, ezért itt nem írunk róluk bővebben.

\subsubsection{Mentés: adatírás}

Az elkészülések után létrehozzunk a \texttt{ui/save.rs} fájlt,
és definiáljuk egy mentés művelet visszatérési lehetőségeit:

\lstinputlisting[firstline=78, lastline=82, language=Rust]{./kodreszletek/02_save.rs}

A \texttt{SaveAction} enumeráció jelzi, hogy a mentés folyamata hogy sikerült:
Lehet, hogy a mentés megbukott vagy sikerült, illetve lehet, hogy sikerült \textit{és}
egy új fájlt hozott létre.

A Rust-ban lehetséges az, hogy egy enumeráció variánsainak adattagokat adunk.
Az adattagok lehetnek egy- vagy több elemű rendezett listában, illetve lehetnek
megnevezett adattagok is -- hasonlóképpen egy struktúrához.
Ehhez hasonló enumerációt már használtunk: Az \texttt{Option<T>} két variánssal térhet vissza,
\texttt{None}, ami megfelel a más nyelvekbeli \texttt{null} értéknek, vagy
\texttt{Some(T)}, ami azt jelzi, hogy tényleges \texttt{T} értéket kaptunk.

Ezt az új enumerációnkat fel is használjuk a következő metódusunkban, amiben
kiírjuk az adatokat a megadott fájlba:

\lstinputlisting[firstline=84, lastline=85, language=Rust]{./kodreszletek/02_save.rs}

Ennél a metódusnál az útvonalat egy \texttt{Option}-ba csomagoljuk be, így lehetővé téve
azt, hogy ha útvonal nélkül hívjuk meg a metódust, akkor az kérdezzen rá, hogy hová
szeretnénk menteni a fájlt.

Továbbá itt megjelenik az \texttt{Option} "testvére," a \texttt{Result}.
Még az előbbi a \texttt{null} értékeket váltja le, az utóbbi a különféle kivételkezelési
módszerek helyett jön be.
A \texttt{Result<T, E>} visszatérhet egy \texttt{Ok(T)} értékkel, ami jelzi,
hogy a műveleteink hibák, illetve kivételek nélkül lefutottak, és \texttt{Some(T)}-be 
becsomagolva visszaadja az elvárt visszatérési értéket,
illetve visszatérhet egy \texttt{Err(E)}-vel, ami esetén becsomagolva kapunk vissza
egy hibastruktúrát, amit aztán később kezelhetünk.

A Rust-ban sok kényelmi funkció létezik, és itt is felhasználunk egyet.
A fenti magyarázat az \texttt{std::Result} struktúrára vonatkozik, viszont mi az
\texttt{std::io::Result} struktúrát használjuk.
A különbség nem drasztikus, az \texttt{std::io::Result<T>} megfelel az \texttt{std::Result<T, std::io::Error>}
struktúrának, így minden bemeneti-kimeneti hibát magában foglalva.

A metódus legelső lépéseként megnézzük, hogy kaptunk-e egy útvonalat, és ha igen,
akkor a megadott útvonalba kiíratjuk a teljes forráskód-szerkesztő puffert,
és visszatérünk egy sikeres mentés enumeráció-variánssal.

\lstinputlisting[firstline=87, lastline=95, language=Rust]{./kodreszletek/02_save.rs}

Itt egyszerre több új Rust-egyedi tulajdonság lép előtérbe.
Az \texttt{if let} szintaxis lecseréli a \texttt{match} szintaxis használatát enumerációk esetén,
ugyanis a Rust-ban minden \texttt{match} ágat le kell kezelni, még azokat is, 
amelyekkel nem szeretnénk semmit sem tenni.
Az \texttt{if let} szintaxis ezen túl ki is csomagolja a megadott enumeráció-variánst, 
ha az tartalmaz egy rendezett listát vagy nevesített adattagokat,
hiszen a fordító biztosan tudja, hogy a programozó mivel dolgozik az \texttt{if let} blokkon belül. 

Továbbá a külső \texttt{path} változót \textit{árnyékoljuk}.
Az árnyékolás olyan esetekben előnyös, ahol egy megadott változón műveleteket hajtunk végre,
de annak értékét nem változtatjuk.
Ilyenkor felesleges más változónevet bevezetni, illetve néha negatív hatása is lehet,
ha az új változónév nem írja le tisztán a változó mivoltját a névkonfliktus miatt.

Tehát, a fenti \texttt{if let} szintaxis azt mondja, hogy ha a metódushívás során
megadtunk egy \texttt{path}-ot, akkor azt kicsomagolva dolgozunk tovább,
ellenkező esetben a metódus későbbi részeivel folytatjuk a program futását.

A könnyebb érthetőség kedvéért tekintsük meg az alábbi három, ekvivalens szintaxist!

\lstinputlisting[firstline=153, lastline=167, language=Rust]{./kodreszletek/02_save.rs}

A legbőbeszédűbb megoldás a már említett \texttt{match} szintaxis.
Ennek több fájdalompontja is van, mint például az összes ág lekezelésének kötelezettsége,
még akkor is, ha ott nem kívánunk semmit tenni, illetve az, 
hogy az ágakban még ki is kell csomagolnunk a \texttt{Some}-ban lévő értéket --
a tényleges érték eléréséhez három behúzás szükséges, 
és ugyanúgy három különböző változónév ugyanazon érték leírásához!

A második megoldásban az \texttt{if let} szintaxissal megspóroljuk a \texttt{match}-csel
járó sablonkód nagy részét -- itt már az \texttt{if let} hatáskörén belül a \texttt{lezetik} a kicsomagolt változó.
Viszont az a probléma, hogy két külön néven utalunk ugyanarra az értékre még mindig fennáll.

Ezért a harmadik megoldásban beárnyékoljuk a \texttt{path} változót.
Az \texttt{if let} hatáskörén belül így a kicsomagolt értéket jelenti, 
azon kívül pedig a becsomagolt értéket.

Ha viszont a metódus nem kap egyáltalán \texttt{path} értéket, 
akkor a második felében felnyitunk a felhasználónak egy fájlválasztó ablakot,
és megkérjük, hogy mentse el a fájlt: 

\lstinputlisting[firstline=97, lastline=108, language=Rust]{./kodreszletek/02_save.rs}

A futása ennek a résznek hasonló, annyi különbséggel, hogy mint a mentési ablak futásától
kapott helyre írjuk a fájlt, és létrehozunk egy új \texttt{ActiveMetadata}-t
a fájl elérési helyével és tartalmával.

Viszont itt azt is ellenőrizzük, hogy mi van akkor, hogy ha a mentési ablak \texttt{None}-nal
tér vissza, azaz nem lett útvonal választva (mert például a felhasználó bezárta az ablakot).
Ebben az esetben a \texttt{Canceled} variánssal térünk vissza.

\subsubsection{Mentés: a mentés metódus}

Az eddigi előkészületeknek köszönhetően a tényleges \texttt{save()} mentés metódus triviálissá válik.
A metódushíváskor kapott útvonalat (vagy annak hiányát) továbbítjuk a \texttt{write\_data()} függvénynek,
és az alapján, hogy mit kaptunk vissza eredményként, vagy egyszerűen csak módosítjuk a jelenlegi
fájl tartalmát (pontosabban, legenerálunk a tartalma alapján egy új hash szummát),
vagy megtesszük azt, és átírjuk a fejléc címét a jelenlegi fájl elérési helyére,
vagy nem teszünk semmit.

Ezek után már csak egy feladatunk marad: a \texttt{ui.rs} fájlba belevinni a modulokat,
illetve azokat publikusan elérhetővé tenni az UI modulon kívül, hogy az \texttt{App}
fel tudja őket használni.